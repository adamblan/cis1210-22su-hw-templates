\documentclass[11pt]{article}
\usepackage{amsmath, amsfonts, amsthm, amssymb}  % Some math symbols
\usepackage{fullpage}

\usepackage[x11names, rgb]{xcolor}
\usepackage{graphicx}
\usepackage{tikz}
\usetikzlibrary{decorations,arrows,shapes}

\usepackage{etoolbox}
\usepackage{enumerate}
\usepackage{listings}

\setlength{\parindent}{0pt}
\setlength{\parskip}{5pt plus 1pt}

\newcommand{\nopagenumbers}{
    \pagestyle{empty}
}

\def\indented#1{\list{}{}\item[]}
\let\indented=\endlist

\providetoggle{questionnumbers}
\settoggle{questionnumbers}{true}
\newcommand{\noquestionnumbers}{
    \settoggle{questionnumbers}{false}
}

\newcounter{questionCounter}
\setcounter{questionCounter}{-1}
\newenvironment{question}[2][\arabic{questionCounter}]{%
    \addtocounter{questionCounter}{1}%
    \setcounter{partCounter}{0}%
    \vspace{.25in} \hrule \vspace{0.4em}%
        \noindent{\bf \iftoggle{questionnumbers}{#1: }{}#2}%
    \vspace{0.8em} \hrule \vspace{.10in}%
}{$ $\newpage}

\newcounter{partCounter}[questionCounter]
\renewenvironment{part}[1][\alph{partCounter}]{%
    \addtocounter{partCounter}{1}%
    \vspace{.10in}%
    \begin{indented}%
       {\bf (#1)} %
}{\end{indented}}


%%%%%%%%%%%%%%%%% Identifying Information %%%%%%%%%%%%%%%%%
%% DO NOT INCLUDE YOUR NAME ANYWHERE IN THE PDF.  WE WANT %%
%% TO GRADE ANONYMOUSLY TO AVOID BIAS!!!!                 %%
%%%%%%%%%%%%%%%%%%%%%%%%%%%%%%%%%%%%%%%%%%%%%%%%%%%%%%%%%%%

%%%%%%%%%%%%%%%%%%% Document Options %%%%%%%%%%%%%%%%%%%%%%
\nopagenumbers
%%%%%%%%%%%%%%%%%%%%%%%%%%%%%%%%%%%%%%%%%%%%%%%%%%%%%%%%%%%

\begin{document}
\begin{question}{Hell-$\mathcal{O}$}
\begin{part}
% Put your solution for part (a) here.
\end{part}

\begin{part}
% Put your solution for part (b) here.
\end{part}

\begin{part}
% Put your solution for part (c) here.
\end{part}

\begin{part}
% Put your solution for part (d) here.
\end{part}
\end{question}

\begin{question}{Seventeen!}
\begin{part}
% Put your solution for part (a) here.
\end{part}

\begin{part}
% Put your solution for part (b) here.
\end{part}

\begin{part}
% Put your solution for part (c) here.
\end{part}

\begin{part}
% Put your solution for part (d) here.
\end{part}
\end{question}

\begin{question}{Is Your Program Running? Better Catch It!}
\begin{part}
% Put your solution for part (a) here.
\end{part}

\begin{part}
% Put your solution for part (b) here.
\end{part}

\begin{part}
% Put your solution for part (c) here.
\end{part}
\end{question}

\begin{question}{Fireworks}
\begin{part}
\textsf{Algorithm:}
% Put your algorithm for this problem here.
\end{part}

\begin{part}
\textsf{Runtime Analysis:}
% Put your runtime analysis for this problem here.
\end{part}

\begin{part}
\textsf{Space Analysis:}
% Put your space analysis for this problem here.
\end{part}
\end{question}

\begin{question}{Slapstick}
\begin{part}
\textsf{Recurrence:}
% Put your recurrence and explanation for this problem here.
\end{part}

\begin{part}
\textsf{Runtime Analysis:}
% Put your runtime analysis for this problem here.
\end{part}
\end{question}

\begin{question}{Pretty Close!}
\begin{part}
\textsf{Recurrence:}
% Put your recurrence and explanation for this problem here.
\end{part}

\begin{part}
\textsf{Runtime Analysis:}
% Put your runtime analysis for this problem here.
\end{part}

\begin{part}
\textsf{Proof of Correctness:}
% Put your proof of correctness analysis for this problem here.
\end{part}
\end{question}
\end{document}
